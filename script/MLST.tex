\section{MLST und MST}

Die Sequenzierung kann verwendet werden, um Unterschiede zwischen den Genomen von Krankheitserregern unterschiedlicher Patienten zu identifizieren und anhand der Anzahl an Mutationen, die die Krankheitserreger voneinander unterscheiden, auf Anteckungsreihenfolgen zu schließen und Krankheitsausbrüche zu rekonstruieren\footnote{Beispielsweise, um den Urpsrung des Ausbruchs zu identifizieren und zu beseitigen - die Wichtigkeit slcher Arbeiten wurde unter Anderem beim EHEC-Ausbruch 2011 deutlich.}. Ohne eine vollständige Rekonstruktion des Genoms des Krankheitserregers, die bis vor ca. 20 Jahren prohibitiv teuer war\footnote{Die Rekonstruktion des ersten menschlichen Genoms mittels Sanger-Sequenzierung kostete 1999-2000 ca. 300 Millionen US-Dollar. Ein bakterielles Genom ist ca. 100-fach kleiner als das Menschliche, eine komplette Rekonstruktion der Genome mehrerer potentiell an einem Ausbruch beteiligter Bakterien würde aber entsprechend immer noch Millionen Euro kosten.}, und heute mit modernen Sequenziermethoden immer noch insbesondere bioinformatisch aufwändig ist, lässt sich aber keine exakte Bestimmung aller Mutationen durchführen. Entsprechend wurde eine andere, auch mittels Sanger einfach durchführbare Methode entwickelt: Das \textbf{Multi-Locus Sequence Typing} (\textbf{MLST}).

\subsection{MLST}

\subsection{cgMLST und wgMLST}

\subsection{MST und Krankheitserreger-Ausbrüche}

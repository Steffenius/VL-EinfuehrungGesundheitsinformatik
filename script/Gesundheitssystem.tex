\section{Grundlagen des Gesundheitssystems}

Das Gesundheitssystem ist ``die Gesamtheit eines organisierten Handelns als Antwort auf das Auftreten von Krankheit und Behinderung und zur Abwehr gesundheitlicher Gefahren''\footnote{Schwartz u. Busse 2003, S. 519}. Da es sehr viele Möglichkeiten gibt, diese Ziele zu erreichen, und den vielen relevanten Faktoren dabei unterschiedliche Wichtigkeiten zugemessen werden können, sind Gesundheitssysteme auf der Welt sehr unterschiedlich ausgeprägt. Beim Gesundheitssystem in Deutschland steht dabei eine Sicherstellung der adäquaten medizinischen Versorgung aller Bürger, die Minimierung der sich aus Krankheit, Behinderung oder Unfall ergebenden Nachteile für den Einzelnen, sowie die individuelle Freiheit (in diesem Kontext insbesondere ausgeprägt in der Freiheit der Wahl der Diestleister für medizinische Versorgung sowie der Krankenkasse) im Fokus. Entsprechend ist eine zentrale Aufgabe des Gesundheitssystems in Deutschland die Sicherstellung einer möglichst gerechten Finanzierung der medizinischen Ausgaben - was nur bei Einhaltung einer kosteneffizienten Arbeitsweise möglich ist.

Zu diesem Zweck arbeiten medizinisches Personal, Krankenkassen, Interessenvertretungen und Behörden in einem komplexen System zusammen. Auf Seiten der Regierung ist das \textbf{Bundesministerium für Gesundheit} (\textbf{BMG}) dabei der wichtigste Akteur. Dieses wird unter Anderem von dem \textbf{Robert Koch-Institut} (\textbf{RKI}) und von dem \textbf{Bundesinstitut für Arzneimittel und Medizinprodukte} (\textbf{BfArM}) beraten. Aufgabe des RKI ist dabei in erster Linie die Forschung an Krankheitserregern, die Unterstützung bei der Aufklärung von Krankheitsausbrüchen und die Untersuchung von Epidemien und Zivilisationskrankheiten. Basierend auf diesen Arbeiten erarbeitet das RKI unter Anderem Empfehlungen zu Notfallplänen, Impfvorsorge\footnote{Ständige Impfkommission}, Krankenhaushygienevorschriften etc., aus denen sich Änderungen an der Gesetzgebung ergeben. Das BfArM befasst sich vor Allem mit der Zulassung und Registrierung von Arzneimitteln sowie der Risikoabschätzung und -Bewertung von Arzneimitteln und Medizinprodukten. Dabei lässt es Ergebnisse aus der eigenen Forschung in diesen Bereichen einfließen. Basierend auf den Ergebnissen seiner Arbeiten unterstützt es ebenfalls das BMG beispielsweise bei der Gesetzgebung oder bei der Abwehr akuter Gefahren für die Gesundheit der Bevölkerung. 

Gemeinsam setzen die Behörden so durch Gesetzgebung und Verordnungen den allgemeinen Rahmen für die medizinische Versorgung. Dieser Rahmen muss sich allerdings außer an den ermittelten Berüfnissen und aktuellen Forschungsergebnissen auch an der Realität des medizinischen Personals, welches die Versorgung durchführt, orientieren. Entsprechend werden viele Entscheidungen über die konkrete Ausgestaltung, wie z.B. über die Zulassung von neuen Untersuchungs- oder Behandlungsmethoden, vom \textbf{gemeinsamen Bundesausschuss} getroffen, dem hochrangige Vertreter der Krankenkassen, Ärzte, Zahnärzte und Psychotherapeuten angehören. Unter Anderem auf Basis dieser Entscheidungen handeln die einzelnen Interessenverbände - die \textbf{Kassenärztliche Bundesvereinigung}, der \textbf{Deutsche Apothekerverband} sowie die \textbf{Deutsche Krankenhausgesellschaft} - Verträge mit dem Interessenverbund der Krankenkassen - dem \textbf{Spitzenverband Bund der Krankenkassen} - bezüglich der Vergütung für die einzelnen Elemente der medizinischen Versorgung aus. 



